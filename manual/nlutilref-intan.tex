% Automatically generated documentation.
\chapter{``nlIntan'' Functions}
\label{sect-intan}

\section{nlIntan\_getAmpChannelFilename.m}

\begin{verbatim}
% function fname = nlIntan_getAmpChannelFilename(indir, bankid, channum)
%
% This returns the data file name for the specified Intan amplifier channel.
% This file doesn't necessarily exist; this just constructs the name.
%
% "indir" is the directory containing Intan data.
% "bankid" is the bank label for the desired channel.
% "channum" is the in-bank channel number for the desired channel.
%
% "fname" is the name of the file that should contain the channel's data.
% This is an empty character array if an error occurred.
\end{verbatim}

\section{nlIntan\_getTimeFilename.m}

\begin{verbatim}
% function fname = nlIntan_getTimeFilename(indir)
%
% This returns the name of the file containing Intan signal timestamps.
% The file doesn't necessarily exist; this just constructs the filename.
% NOTE - Intan saves the sample indices, not an actual time values.
%
% "indir" is the directory to search.
%
% "fname" is the name of the file that should contain sample time indices.
\end{verbatim}

\section{nlIntan\_helper\_probeChannels.m}

\begin{verbatim}
% function [ chandetect chanfiles ] = ...
%   nlIntan_helper_probeChannels(chantest, banklist, chanrange, fnamefunc)
%
% This checks for the existence of data files from specified I/O banks, and
% and probes for channels and banks if asked to do so.
%
% "chantest" is a structure with field names that are bank identifiers
%   ('A', 'B', 'DIN', etc), with each field containing an array of channel
%   numbers to test. See "CHANLIST.txt" for details.
%   An empty structure means "auto-detect all banks". An empty channel array
%   means "auto-detect all channels for this bank".
% "bankids" is a cell array containing a list of bank IDs that are
%   potentially probed.
% "chanrange" is an array of channel numbers that are potentially probed.
% "fnamefunc" points to a function that constructs a filename when passed
%   a bank ID and a channel number as arguments.
%
% "chandetect" is a structure with the same format as "chantest", enumerating
%   the banks and channels from which data was read. Fields in "chantest"
%   that are not "potentially probed" bank identifiers are copied as-is.
% "chanfiles" is an array of structures containing the following fields:
%    "bank"  - Bank identifier (field name).
%    "chan"  - Channel number.
%    "fname" - Name of the file containing this bank/channel's data.
\end{verbatim}

\section{nlIntan\_probeAmpChannels.m}

\begin{verbatim}
% function [ chandetect ampdetect chanfiles ] = ...
%   nlIntan_probeAmpChannels(indir, chantest)
%
% This checks for the existence of data files from specified amplifier
% channels, and probes for channels and amplifiers if asked to do so.
%
% "indir" is the directory to search.
% "chantest" is a structure with field names that are amplifier identifiers
%   ('A', 'B', etc), with each field containing an array of channel numbers
%   to fetch. An empty structure means "auto-detect all amplifiers". An
%   empty array in a channel field means "auto-detect all channels".
%
% "chandetect" is a structure with the same format as "chantest", enumerating
%   the amplifiers and channels from which data was read. Fields in "chantest"
%   that are not amplifier identifiers (per CHANLIST.txt) are copied as-is.
% "ampdetect" is a cell array of field names, containing only detected
%   amplifier IDs.
% "chanfiles" is an array of structures containing the following fields:
%   "bank"  - Amplifier identifier string.
%   "chan"  - Channel number.
%   "fname" - Name of the file containing this bank/channel's data.
\end{verbatim}

\section{nlIntan\_readAmpChannels.m}

\begin{verbatim}
% function [is_ok chandetect ampdetect timedata chandata] = ...
%   nlUtil_readIntanAmpChannels(indir, chanlist)
%
% This attempts to read individual Intan amplifier channel data files from
% the specified directory. The time series is also read.
%
% "indir" is the directory to search.
% "chanlist" is a structure with field names that are amplifier identifiers
%   ('A', 'B', etc), with each field containing an array of channel numbers
%   to fetch. An empty structure means "auto-detect all amplifiers". An
%   empty array in a channel field means "auto-detect all channels".
%
% "is_ok" is true if data was read and false otherwise.
% "chandetect" is a structure with the same format as "chanlist", enumerating
%   the amplifiers and channels from which data was read.
% "timedata" contains the time series (in samples, not seconds).
% "chandata" is an array of structures containing the following fields:
%   "bank" - Amplifier identifier string.
%   "chan" - Channel number.
%   "fname" - Filename
%   "data" - Sample data series.
\end{verbatim}

\section{nlIntan\_readMetadata.m}

\begin{verbatim}
% function [ is_ok metadata ] = nlUtil_readIntanMetadata(fname)
%
% This attempts to read selected parts of the specified Intan metadata file.
% If successful, "is_ok" is set to "true" and "metadata" is a structure with
% the following fields:
%
% "devtype"  - "RHS" for a recording controller, "RHD" for stimulation.
% "samprate" - Sampling rate in Hz.
%
% On failure, "is_ok" is set to "false" and "metadata" is an empty structure.
%
% FIXME - Ignoring most of the metadata. Use Intan's functions if you need
% all of it.
\end{verbatim}

% This is the end of the file.
