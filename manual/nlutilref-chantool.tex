% Automatically generated documentation.
\chapter{``nlChan'' Functions}
\label{sect-chan}

\section{nlChan\_applyArtifactReject.m}

\begin{verbatim}
% function [ newdata fracbad ] = nlChan_applyArtifactReject( ...
%   wavedata, samprate, tuningparams, keepnan )
%
% This performs truncation and artifact rejection, optionally followed by
% interpolation in the former artifact regions.
%
% "wavedata" is the waveform to process.
% "refdata" is a reference to subtract from the waveform, or [] for no
%   reference. The reference should already be truncated and have artifacts
%   removed, but should retain NaN values to avoid introducing new artifacts.
% "samprate" is the sampling rate.
% "tuningparams" is a structure containing tuning parameters for artifact
%   rejection.
% "keepnan" is true if NaN values are to remain and false if interpolation
%   is to be performed to remove them.
%
% "newdata" is the series after artifact removal.
% "fracbad" is the fraction of samples discarded as artifacts (0..1).
\end{verbatim}

\section{nlChan\_applyFiltering.m}

\begin{verbatim}
% function [ lfpseries spikeseries ] = nlChan_applyFiltering( ...
%   wavedata, samprate, tuningparams );
%
% This performs filtering to suppress power line noise, zero-average the
% signal, and to split the signal into LFP and spike components.
%
% Power line noise filtering and DC removal filtering can be suppressed by
% setting their respective filter frequencies to 0 Hz.
%
% "wavedata" is the waveform to process.
% "samprate" is the sampling rate.
% "tuningparams" is a structure containing tuning parameters for filtering.
%
% "newdata" is the series after filtering.
\end{verbatim}

\section{nlChan\_applySpectSkewCalc.m}

\begin{verbatim}
% function [ spectfreqs spectmedian spectiqr spectskew ] = ...
%   nlChan_applySpectSkewCalc( wavedata, samprate, tuningparams, perclist )
%
% This calls nlProc_calcSpectrumSkew() to compute a persistence spectrum for
% the specified series and to compute statistics and skew for each frequency
% bin.
%
% "wavedata" is the waveform to process.
% "samprate" is the sampling rate.
% "tuningparams" is a structure containing tuning parameters for persistence
%   spectrum generation.
% "perclist" is an array of percentile values that define the tails for
%   skew calculation, per nlProc_calcSkewPercentile().
%
% "spectfreqs" is an array of bin center frequencies.
% "spectmedian" is an array of per-frequency median power values.
% "spectiqr" is an array of per-frequency power interquartile ranges.
% "spectskew" is a cell array, with one cell per "perclist" value. Each cell
%   contains an array of per-frequency skew values.
\end{verbatim}

\section{nlChan\_getArtifactDefaults.m}

\begin{verbatim}
% function paramstruct = nlChan_getArtifactDefaults()
%
% This returns a structure containing reasonable default tuning parameters
% for artifact rejection.
%
% Parameters that will most often be varied are "ampthresh", "diffthresh",
% "trimstart", and "trimend".
\end{verbatim}

\section{nlChan\_getFilterDefaults.m}

\begin{verbatim}
% function paramstruct = nlChan_getFilterDefaults()
%
% This returns a structure containing reasonable default tuning parameters
% for signal filtering.
%
% Parameters that will most often be varied are "powerfreq" and "lfprate".
\end{verbatim}

\section{nlChan\_getPercentDefaults.m}

\begin{verbatim}
% function paramstruct = nlChan_getPercentDefaults()
%
% This returns a structure containing reasonable default tuning parameters
% for spike and burst identification via percentile binning.
%
% Parameters that will most often be varied are "burstselectidx" and
% "spikeselectidx".
\end{verbatim}

\section{nlChan\_getSpectrumDefaults.m}

\begin{verbatim}
% function paramstruct = nlChan_getSpectrumDefaults()
%
% This returns a structure containing reasonable default tuning parameters
% for persistence spectrum generation.
\end{verbatim}

\section{nlChan\_iterateChannels.m}

\begin{verbatim}
% function outdata = ...
%   nlChan_iterateChannels(chanfiles, bankrefs, samprate, refparams, procfunc)
%
% This iterates through a list of channel records, loading and preprocessing
% each channel and then calling a processing function with the channel
% data. Processing output is aggregated and returned.
%
% "chanfiles" is an array of channel file records in the format returned by
%   nlIntan_probeAmpChannels(). These have the following fields:
%   "bank" - Amplifier identifier string.
%   "chan" - Channel number.
%   "fname" - Name of the file containing this bank/channel's data.
% "bankrefs" is a structure with field names that are bank identifiers, with
%   each field containing a single channel number specifying the in-bank
%   channel to use as a reference for the remaining channels. If an empty
%   array is present or if a bank identifier is absent, no reference is used
%   for that bank. These channels must also be present in "chanfiles".
% "samprate" is the sampling rate.
% "tuningart" is a structure containing tuning parameters for artifact
%   rejection.
% "procfunc" is a function handle that is called per file. It has the form:
%     resultval = procfunc(chanrec, wavedata)
%   This is typically an anonymous function that wraps a function with
%   additional arguments.
%
% "outdata" is a copy of "chanfiles" augmented with an additional "result"
%   field, containing "resultval" returned from procfunc(). Only records
%   corresponding to channels that were processed are present; channels that
%   were discarded due to artifacts or that were references are absent.
\end{verbatim}

\section{nlChan\_processChannel.m}

\begin{verbatim}
% function resultstats = nlChan_processChannel( wavedata, samprate, ...
%   tuningfilt, tuningspect, tuningperc )
%
% This accepts a wideband waveform, performs filtering to split it into
% spike and LFP signals, and calculates various statistics for each of these
% signals.
%
% This is intended to be called by nlChan_iterateChannels() via a wrapper.
%
% "wavedata" is the waveform to process.
% "samprate" is the sampling rate.
% "tuningfilt" is a structure containing tuning parameters for filtering.
% "tuningspect" is a structure containing tuning parameters for persistence
%   spectrum generation.
% "tuningperc" is a structure containing tuning parameters for spike and
%   burst identification via percentile binning.
%
% "resultstats" is a structure containing the following fields:
%   "spikemedian", "spikeiqr", "spikeskew", and "spikepercentvals" are the
%     corresponding fields returned by nlProc_calcSkewPercentile() using the
%     high-pass-filtered spike signal.
%   "spikebincounts" and "spikebinedges" are the the corresponding fields
%     returned by histcounts() using a normalized version of the spike signal.
%   "spectfreqs", "spectmedian", "spectiqr", and "spectskew" are the
%     corresponding fields returned by nlChan_applySpectSkewCalc() using the
%     low-pass-filtered LFP signal.
%   "persistvals", "persistfreqs", and "persistpowers" are the corresponding
%     fields returned by pspectrum() using the LFP signal.
\end{verbatim}

\section{nlChan\_rankChannels.m}

\begin{verbatim}
% function [ bestlist typbest typmiddle typworst ] = ...
%   nlChan_rankChannels( chanrecs, maxperbank, typfrac, scorefunc )
%
% This process a list of channel records returned by nlChan_iterateChannels().
% Channel records receive a score, and the list is sorted by that score.
% Statistics for typical records and aggregate statistics are returned.
% The sorted list is pruned to include at most a certain number of channels
% per bank, and is then returned.
% NOTE - The "typical" records are not necessarily in the pruned result list.
%
% "chanrecs" is the list of channel statistics records to process.
% "maxperbank" is the maximum number of channels per bank in the returned list.
% "typfrac" is the percentile for finding "typical" good and bad records.
%   This is a number between 0 and 50 (typically 5, 10, or 25).
% "scorefunc" is a function handle that is called for each channel. It has
%   the form:
%     scoreval = scorefunc(resultval)
%   The "resultval" argument is the chanrecs(n).result field, per
%   nlChan_iterateChannels().
%   Higher scores are better, for purposes of this function. A score of NaN
%   squashes a result (removing it from the result list).
%
% "bestlist" is a subset of the sorted channel record list containing the
%   highest-scoring entries subject to the constraints described above.
% "typbest" is the channel record for the top Nth percentile channel.
% "typmiddle" is the channel record for the median channel.
% "typworst" is the channel record for the bottom Nth percentile channel.
\end{verbatim}

% This is the end of the file.
