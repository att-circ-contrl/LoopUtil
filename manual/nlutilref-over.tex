% Neuroloop utilities - Library reference - Overview
% Written by Christopher Thomas.

\chapter{Overview}
\label{sect-over}

The NeuroLoop utility library functions are divided into several categories:

\begin{itemize}

\item \textbf{``Processing''} functions (ch. \ref{sect-proc}) perform signal
processing operations on data series such as filtering, artifact rejection,
statistical calculations, and so forth.

\item \textbf{``Utility''} functions (ch. \ref{sect-util}) perform
miscellaneous operations that are not covered by the other categories.

\item \textbf{``Plotting''} functions (ch. \ref{sect-plot}) render plots
of various types to files, figures, or axes. These are included as an aid
to rapid prototyping, and are used by the GUI scripts. The output is
generally not publication-ready.

\item \textbf{``I/O''} functions (ch. \ref{sect-io}) facilitate operations
for reading and writing data that aren't vendor-specific.

\item \textbf{``Intan''} functions (ch. \ref{sect-intan}) facilitate the use
of datasets stored in Intan's format.

\item \textbf{``Vendor-Supplied Intan''} functions
(ch. \ref{sect-vend-intan}) are functions derived from vendor-supplied code
that are in turn wrapped by the functions in ch. \ref{sect-intan}.

\item \textbf{``Channel Tool''} functions (ch. \ref{sect-chan}) perform
operations used by the ``Channel Tool'' utility. The intention is that all
operations that are not tied to GUI implementation are packaged as library
functions for reuse outside of that application.

\end{itemize}

Several types of structure and several types of function handle are used by
the library functions. These are described in Chapter \ref{sect-notes}.

Sample code is provided in Chapter \ref{sect-examples}.

%
% This is the end of the file.
