% Automatically generated documentation.
\chapter{``nlProc'' Functions}
\label{sect-proc}

\section{nlProc\_calcSkewPercentile.m}

\begin{verbatim}
% function [ seriesmedian seriesiqr seriesskew rawpercentiles ] = ...
%   nlProc_calcSkewPercentile(dataseries, tailpercent)
%
% This computes the median and the (tailpercent, 100%-tailpercent) tail
% percentiles for the specified series, and evaluates skew by comparing the
% midsummary (average of the tail values) with the median. The result is
% normalized (a skew of +/- 1 is a displacement by +/- the interquartile
% range).
%
% "dataseries" is the sample sequence to process.
% "tailpercent" is an array of tail values to test.
%
% "seriesmedian" is the series median value.
% "seriesiqr" is the series interquartile range.
% "seriesskew" is an array of normalized skew values corresponding to the
%   tail percentages.
% "rawpercentiles" is an array with the actual percentile values used for
%   skew calculations. It contains percentile values corresponding to
%   [ (tailpercent) (median) (100 - tailpercent) (25%) (75%) ].
\end{verbatim}

\section{nlProc\_calcSpectrumSkew.m}

\begin{verbatim}
% function [ spectfreqs spectmedian spectiqr spectskew ] = ...
%   nlProc_calcSpectrumSkew( dataseries, samprate, ...
%   freqrange, freqperdecade, wintime, winsteps, tailpercent)
%
% This computes a persistence spectrum for the specified series, and finds
% the median, interquartile range, and the normalized skew for each frequency
% bin. "Skew" is defined per nlProc_calcSkewPercentile().
%
% "dataseries" is the data series to process.
% "samprate" is the sampling rate of the data series.
% "freqrange" [ fmin fmax ] specifies the frequency band to evaluate.
% "freqperdecade" is the number of frequency bins per decade.
% "wintime" is the window duration in seconds to compute the time-windowed
%   Fourier transform with.
% "winsteps" is the number of overlapping steps taken when advancing the time
%   window. The window advances by wintime/winsteps seconds per step.
% "tailpercent" is an array of percentile values that define the tails for
%   skew calculation, per nlProc_calcSkewPercentile().
%
% "spectfreqs" is an array of bin center frequencies.
% "spectmedian" is an array of per-frequency median power values.
% "spectiqr" is an array of per-frequency power interquartile ranges.
% "spectskew" is a cell array, with one cell per "tailpercent" value. Each
%   cell contains an array of per-frequency skew values.
\end{verbatim}

\section{nlProc\_fillNaN.m}

\begin{verbatim}
% function newseries = nlProc_fillNaN( oldseries )
%
% This interpolates NaN segments within the series using linear interpolation,
% and then fills in NaNs at the end of the series by replicating samples.
% This makes the derivative discontinuous when filling endpoints but prevents
% large excursions from curve fit extrapolation.
%
% "oldseries" is the series containing NaN segments.
%
% "newseries" is the interpolated series without NaN segments.
\end{verbatim}

\section{nlProc\_filterSignal.m}

\begin{verbatim}
% function [ lfpseries spikeseries ] = nlProc_filterSignal( oldseries, ...
%   samprate, lfprate, lowpassfreq, powerfreq, dcfreq )
%
% This applies several filters:
% - A DC removal filter.
% - A notch filter to remove power line noise.
% - A low-pass filter to isolate the local field potential signal.
% The full-rate LFP series is subtracted from the original series to produce
% a high-pass-filtered "spike" series, and a downsampled LFP series is
% also returned.
%
% "oldseries" is the original wideband signal.
% "samprate" is the sampling rate of the wideband signal.
% "lfprate" is the desired sampling rate of the LFP signal. This should
%   cleanly divide "samprate" (samprate = k * lfprate for some integer k).
% "lowpassfreq" is the edge of the pass-band for the LFP signal. This is
%   lower than the filter's corner frequency; it's the 0.2 dB frequency.
% "powerfreq" is an array of values specifying the center frequencies of the
%   power line notch filter. This is typically 60 Hz or 50 Hz (a single
%   value), but may contain multiple values to filter harmonics. An empty
%   array disables this filter.
% "dcfreq" is the edge of the pass-band for the high-pass DC removal filter.
%   Set to 0 to disable this filter. This is higher than the corner frequency;
%   it's the 0.2 dB frequency (ripple is flat above it).
%
% "lfpseries" is the downsampled low-pass-filtered signal.
% "spikeseries" is the full-rate high-pass-filtered signal.
%
% Filters used are IIR, called with "filtfilt" to remove time offset by
% running the filter forwards and backwards in time. The power line filter
% takes about 1/2 second to fully stabilize, and the low-pass LFP filter takes
% about 1/2 period to 1 period to fully stabilize. Edge effects may occur
% within this distance of the start and end of the signal.
% The DC rejection filter also takes at least 1 period to stabilize. Since
% it's applied in both directions, and won't perturb the pass-band, it should
% be well-behaved over the entire signal.
%
% The LFP sampling rate should be at least 10 times "lowpassfreq" to avoid
% aliasing during downsampling. The DC rejection filter pass frequency
% should be no lower than half the lowest frequency of interest, to
% minimize edge effects.
\end{verbatim}

\section{nlProc\_removeArtifactsSigma.m}

\begin{verbatim}
% function newseries = nlProc_removeArtifactsSigma( oldseries, ...
%   ampthresh, derivthresh, ampthreshfall, derivthreshfall, ...
%   trimbefore, trimafter, smoothsamps, dcsamps )
%
% This identifies artifacts as excursions in the signal's amplitude or
% derivative, and replaces affected regions with NaN. Excursion thresholds
% are expressed in terms of the standard deviation of the signal or its
% derivative.
%
% "oldseries" is the series to process.
% "ampthresh" is the threshold for flagging amplitude excursion artifacts.
% "derivthresh" is the threshold for flagging derivative excursion artifacts.
% "ampthreshfall" is the turn-off threshold for amplitude artifacts.
% "derivthreshfall" is the turn-off threshold for derivative artifacts.
% "trimbefore" is the number of samples to squash ahead of the artifact.
% "trimafter" is the number of samples to squash after the artifact.
% "smoothsamps" is the size of the smoothing window to apply before taking
%   the derivative, or 0 for no smoothing.
% "dcsamps" is the size of the window for computing local DC average removal
%   ahead of computing signal statistics.
%
% Regions where the amplitude or derivative exceeds the threshold are flagged
% as artifacts. These regions are widened to encompass the region where the
% amplitude or derivative exceeds the "fall" threshold, and then padded by
% the specified number of samples. This is intended to correctly handle
% square-pulse artifacts and fast-step-slow-decay artifacts.
\end{verbatim}

\section{nlProc\_removeTimeRanges.m}

\begin{verbatim}
% function newseries = ...
%   nlProc_removeTimeRanges( oldseries, samprate, trimtimes )
%
% This NaNs out specified regions of the input signal.
%
% "oldseries" is the series to process.
% "samprate" is the sampling rate of the input signal.
% "trimtimes" is a cell array containing time spans to NaN out. Time spans
%   have the form "[ time1 time2 ]", where times are in seconds. Negative
%   times are relative to the end of the signal, positive times are relative
%   to the start of the signal (both start at 0 seconds). Use a very small
%   negative value for "-0".
%
% "newseries" is a modified version of the input series with the specified
%   time ranges set to NaN.
\end{verbatim}

\section{nlProc\_trimEndpoints.m}

\begin{verbatim}
% function newseries = ...
%   nlProc_trimEndpints( oldseries, samprate, trimstart, trimend )
%
% This crops the specified durations from the start and end of the supplied
% signal.
%
% "oldseries" is the series to process.
% "samprate" is the sampling rate of the input signal.
% "trimstart" is the number of seconds to remove from the beginning.
% "trimend" is the number of seconds to remove from the end.
%
% "newseries" is a truncated version of the input signal.
\end{verbatim}

% This is the end of the file.
