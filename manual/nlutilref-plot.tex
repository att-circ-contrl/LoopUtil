% Automatically generated documentation.
\chapter{``nlPlot'' Functions}
\label{sect-plot}

\section{nlPlot\_axesPlotExcursions.m}

\begin{verbatim}
% function nlPlot_axesPlotExcursions( thisax, ...
%   spectfreqs, spectmedian, spectiqr, spectskew, percentlist, ...
%   want_relative, figtitle )
%
% This plots LFP power excursions, either as relative power excess alone or
% against the median power spectrum. See nlChan_applySpectSkewCalc() for
% details of skew calculation and array contents.
% The plot is rendered to the specifed set of figure axes.
%
% "thisax" is the "axes" object to render to.
% "spectfreqs" is an array of frequency bin center frequencies.
% "spectmedian" is an array of per-frequency median power values.
% "spectiqr" is an array of per-frequency power interquartile ranges.
% "spectskew" is a cell array, with one cell per "percentlist" value. Each
%   cell contains an array of per-frequency skew values.
% "percentlist" is an array of percentile values that define the tails for
%   skew calculations, per nlProc_calcSkewPercentile().
% "want_relative" is true to plot relative power excess alone, and false to
%   plot against the median power spectrum.
% "figtitle" is the title to use for the figure, or '' for no title.
\end{verbatim}

\section{nlPlot\_axesPlotPersist.m}

\begin{verbatim}
% function nlPlot_plotPersist( thisax, ...
%   persistvals, persistfreqs, persistpowers, want_log, figtitle )
%
% This plots a pre-tabulated persistence spectrum. See "pspectrum()" for
% details of input array structure.
%
% "thisax" is the "axes" object to render to.
% "thisfig" is the figure to render to (this may be a UI component).
% "persistvals" is the matrix of persistence spectrum fraction values.
% "persistfreqs" is the list of frequencies used for binning.
% "persistpowers" is the list of power magnitudes used for binning.
% "want_log" is true if the frequency axis should be plotted on a log scale
%   (it's computed on a linear scale).
% "figtitle" is the title to use for the figure, or '' for no title.
\end{verbatim}

\section{nlPlot\_axesPlotSpikeHist.m}

\begin{verbatim}
% function nlPlot_plotSpikeHist( thisax, ...
%   bincounts, binedges, percentamps, percentpers, figtitle )
%
% This plots a pre-tabulated histogram of normalized spike waveform
% amplitude. For channels with real spikes, tails are asymmetrical.
%
% "thisax" is the "axes" object to render to.
% "bincounts" is an array containing bin count values, per histogram().
% "binedges" is an array containing the histogram bin edges, per histogram().
% "percentamps" is an array of normalized amplitudes corresponding to desired
%   tail percentiles to highlight. Entries 1..N are tail percentile amplitudes,
%   entry N+1 is the median, and entries N+2..2N+1 are (100%-tail) amplitudes.
% "percentpers" is an array naming desired tail percentiles to highlight.
% "figtitle" is the title to use for the figure, or '' for no title.
\end{verbatim}

\section{nlPlot\_plotExcursions.m}

\begin{verbatim}
% function nlPlot_plotExcursions( thisfig, oname, ...
%   spectfreqs, spectmedian, spectiqr, spectskew, percentlist, ...
%   want_relative, figtitle )
%
% This plots LFP power excursions, either as relative power excess alone or
% against the median power spectrum. See nlChan_applySpectSkewCalc() for
% details of skew calculation and array contents.
%
% "thisfig" is the figure to render to (this may be a UI component).
% "oname" is the filename to save to, or '' to not save.
% "spectfreqs" is an array of frequency bin center frequencies.
% "spectmedian" is an array of per-frequency median power values.
% "spectiqr" is an array of per-frequency power interquartile ranges.
% "spectskew" is a cell array, with one cell per "percentlist" value. Each
%   cell contains an array of per-frequency skew values.
% "percentlist" is an array of percentile values that define the tails for
%   skew calculations, per nlProc_calcSkewPercentile().
% "want_relative" is true to plot relative power excess alone, and false to
%   plot against the median power spectrum.
% "figtitle" is the title to use for the figure, or '' for no title.
\end{verbatim}

\section{nlPlot\_plotPersist.m}

\begin{verbatim}
% function nlPlot_plotPersist( thisfig, oname, ...
%   persistvals, persistfreqs, persistpowers, want_log, figtitle )
%
% This plots a pre-tabulated persistence spectrum. See "pspectrum()" for
% details of input array structure.
%
% "thisfig" is the figure to render to (this may be a UI component).
% "oname" is the filename to save to, or '' to not save.
% "persistvals" is the matrix of persistence spectrum fraction values.
% "persistfreqs" is the list of frequencies used for binning.
% "persistpowers" is the list of power magnitudes used for binning.
% "want_log" is true if the frequency axis should be plotted on a log scale
%   (it's computed on a linear scale).
% "figtitle" is the title to use for the figure, or '' for no title.
\end{verbatim}

\section{nlPlot\_plotSpikeHist.m}

\begin{verbatim}
% function nlPlot_plotSpikeHist( thisfig, oname, ...
%   bincounts, binedges, percentamps, percentpers, figtitle )
%
% This plots a pre-tabulated histogram of normalized spike waveform
% amplitude. For channels with real spikes, tails are asymmetrical.
%
% "thisfig" is the figure to render to (this may be a UI component).
% "oname" is the filename to save to, or '' to not save.
% "bincounts" is an array containing bin count values, per histogram().
% "binedges" is an array containing the histogram bin edges, per histogram().
% "percentamps" is an array of normalized amplitudes corresponding to desired
%   tail percentiles to highlight. Entries 1..N are tail percentile amplitudes,
%   entry N+1 is the median, and entries N+2..2N+1 are (100%-tail) amplitudes.
% "percentpers" is an array naming desired tail percentiles to highlight.
% "figtitle" is the title to use for the figure, or '' for no title.
\end{verbatim}

% This is the end of the file.
